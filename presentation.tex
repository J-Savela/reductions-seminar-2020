%%%%%%%%%%%%%%%%%%%%%%%%%%%%%%%%%%%%%%%%%
% Beamer Presentation
% LaTeX Template
% Version 1.0 (10/11/12)
%
% This template has been downloaded from:
% http://www.LaTeXTemplates.com
%
% License:
% CC BY-NC-SA 3.0 (http://creativecommons.org/licenses/by-nc-sa/3.0/)
%
%%%%%%%%%%%%%%%%%%%%%%%%%%%%%%%%%%%%%%%%%

%----------------------------------------------------------------------------------------
%	PACKAGES AND THEMES
%----------------------------------------------------------------------------------------

\documentclass[handout]{beamer}
%\documentclass{beamer}

\usepackage[utf8]{inputenc}
\usepackage{csquotes}

\mode<presentation> {

% The Beamer class comes with a number of default slide themes
% which change the colors and layouts of slides. Below this is a list
% of all the themes, uncomment each in turn to see what they look like.

%\usetheme{default}
%\usetheme{AnnArbor}
%\usetheme{Antibes}
%\usetheme{Bergen}
%\usetheme{Berkeley}
%\usetheme{Berlin}
%\usetheme{Boadilla}
%\usetheme{CambridgeUS}
\usetheme{Copenhagen}
%\usetheme{Darmstadt}
%\usetheme{Dresden}
%\usetheme{Frankfurt}
%\usetheme{Goettingen}
%\usetheme{Hannover}
%\usetheme{Ilmenau}
%\usetheme{JuanLesPins}
%\usetheme{Luebeck}
%\usetheme{Madrid}
%\usetheme{Malmoe}
%\usetheme{Marburg}
%\usetheme{Montpellier}
%\usetheme{PaloAlto}
%\usetheme{Pittsburgh}
%\usetheme{Rochester}
%\usetheme{Singapore}
%\usetheme{Szeged}
%\usetheme{Warsaw}

% As well as themes, the Beamer class has a number of color themes
% for any slide theme. Uncomment each of these in turn to see how it
% changes the colors of your current slide theme.

%\usecolortheme{albatross}
%\usecolortheme{beaver}
%\usecolortheme{beetle}
%\usecolortheme{crane}
%\usecolortheme{dolphin}
%\usecolortheme{dove}
%\usecolortheme{fly}
%\usecolortheme{lily}
%\usecolortheme{orchid}
%\usecolortheme{rose}
%\usecolortheme{seagull}
%\usecolortheme{seahorse}
%\usecolortheme{whale}
%\usecolortheme{wolverine}

\addtobeamertemplate{navigation symbols}{}{%
    \usebeamerfont{footline}%
    \usebeamercolor[fg]{footline}%
    \hspace{1em}%
    \insertframenumber/\inserttotalframenumber
}
%\setbeamertemplate{footline} % To remove the footer line in all slides uncomment this line
%\setbeamertemplate{footline}[page number] % To replace the footer line in all slides with a simple slide count uncomment this line

%\setbeamertemplate{navigation symbols}{} % To remove the navigation symbols from the bottom of all slides uncomment this line
}

\usepackage{graphicx} % Allows including images
\usepackage{booktabs} % Allows the use of \toprule, \midrule and \bottomrule in tables
\usepackage{epstopdf}
\usepackage[backend=bibtex, style=numeric-comp, isbn=false, url=false, hyperref=true]{biblatex}
\usepackage{bibentry}
\usepackage[algoruled]{algorithm2e}
\setbeamertemplate{bibliography item}{\insertbiblabel}

\renewcommand*{\bibfont}{\tiny}

%----------------------------------------------------------------------------------------
%	TITLE PAGE
%----------------------------------------------------------------------------------------

\title[]{SAT encodings for hard problems in computational biology} % The short title appears at the bottom of every slide, the full title is only on the title page

\author{Jarkko Savela} % Your name
\institute{University of Helsinki} % Your institution as it will appear on the bottom of every slide, may be shorthand to save space
\date{\today} % Date, can be changed to a custom date
\bibliography{presentation}
\begin{document}

\begin{frame}
\titlepage % Print the title page as the first slide
\end{frame}

\begin{frame}
\begin{itemize}
\item \fullcite{DBLP:conf/alcob/BrownZG20}
\end{itemize}
\end{frame}

\begin{frame}
\frametitle{Background}
\begin{itemize}
\item ILP has been very successful in computational biology~\cite{gusfield2019integer}
\item BUT ILP is ineffective for certain problems
\item SAT has been successfully applied to certain computational biology problems such as haplotyping by pure parsimony~\cite{DBLP:conf/aaai/LynceM06, DBLP:journals/anor/GracaMLO11, DBLP:journals/jda/JagerCZ16}
\item Try other techniques (such as SAT) on problems ILP cannot handle
\end{itemize}
\end{frame}

\begin{frame}
\frametitle{Article contents}
\begin{itemize}
\item Empirical comparison of SAT-encodings and ILP formulations of 4 problems from computational biology
\begin{enumerate}
\item Protein folding under HP model
\item Transforming gene orders by reversals
\item Computing the history bound
\item Haplotyping by pure parsimony
\end{enumerate}
\item SAT solver used is pLingeling and Glucose, ILP solver Gurobi
\item ILP was better for problem 1 with some exceptions
\item SAT was superior in solving problems 2, 3 and 4 (wrt runtime and avoiding extreme behavior of ILP solvers)
\item SAT was able to solve instances of problem 4 that ILP could not solve
\end{itemize}
\end{frame}

\begin{frame}
\frametitle{HP model}
\begin{itemize}
\item Proteins are linear chains of amino acids
\item Each amino acid is hydrophilic (H) or hydrophobic (P)
\item Hydrophilic amino acids form hydrogen bonds 
\item Prototein is a binary sequence with 0 encoding P and 1 encoding H
\item Problem: Embed a given prototein into a $2$-D (or $3$-D) grid
\item Objective: maximize adjacent H's (i.e., hydrogen bonds)
\end{itemize}
\end{frame}

\begin{frame}
\frametitle{SAT encoding}
\begin{itemize}
\item Binary sequence $S$, grid $G$
\item Variables:
\begin{itemize}
\item[$X_{i,j}$]: $S[i]$ is at position $j$ of $G$
\item [$T_j$]: true iff a $1$ occupies position $j$ of $G$
\item [$C_{j,m}$]: true iff both positions $j$ and $m$ are occupied by a $1$
\end{itemize}
\item Constraints
\begin{itemize}
\item Each grid point has at most one sequence entry
\item ech sequence entry is in exactly one grid point
\item Adjacent sequence entries are adjacent in the grid
\item + Machinery for counting number of contacts
\end{itemize}
\item Optimization via binary search
\item ILP achieved better runtimes but took long in some instances
\item SAT had slightly longer runtimes but they performance was more consistent
\end{itemize}
\end{frame}

\begin{frame}
\frametitle{Direction for my seminar report}
\begin{itemize}
\item More in-depth exploration of the use of SAT in computational and systems biology
\item (+comparison with ILP)
\end{itemize}
\end{frame}

\begin{frame}[allowframebreaks]
\frametitle{References}
\printbibliography
\end{frame}


\end{document} 