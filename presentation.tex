%%%%%%%%%%%%%%%%%%%%%%%%%%%%%%%%%%%%%%%%%
% Beamer Presentation
% LaTeX Template
% Version 1.0 (10/11/12)
%
% This template has been downloaded from:
% http://www.LaTeXTemplates.com
%
% License:
% CC BY-NC-SA 3.0 (http://creativecommons.org/licenses/by-nc-sa/3.0/)
%
%%%%%%%%%%%%%%%%%%%%%%%%%%%%%%%%%%%%%%%%%

%----------------------------------------------------------------------------------------
%	PACKAGES AND THEMES
%----------------------------------------------------------------------------------------

\documentclass[handout]{beamer}
%\documentclass{beamer}

\usepackage[utf8]{inputenc}
\usepackage{csquotes}

\mode<presentation> {

% The Beamer class comes with a number of default slide themes
% which change the colors and layouts of slides. Below this is a list
% of all the themes, uncomment each in turn to see what they look like.

%\usetheme{default}
%\usetheme{AnnArbor}
%\usetheme{Antibes}
%\usetheme{Bergen}
%\usetheme{Berkeley}
%\usetheme{Berlin}
%\usetheme{Boadilla}
%\usetheme{CambridgeUS}
\usetheme{Copenhagen}
%\usetheme{Darmstadt}
%\usetheme{Dresden}
%\usetheme{Frankfurt}
%\usetheme{Goettingen}
%\usetheme{Hannover}
%\usetheme{Ilmenau}
%\usetheme{JuanLesPins}
%\usetheme{Luebeck}
%\usetheme{Madrid}
%\usetheme{Malmoe}
%\usetheme{Marburg}
%\usetheme{Montpellier}
%\usetheme{PaloAlto}
%\usetheme{Pittsburgh}
%\usetheme{Rochester}
%\usetheme{Singapore}
%\usetheme{Szeged}
%\usetheme{Warsaw}

% As well as themes, the Beamer class has a number of color themes
% for any slide theme. Uncomment each of these in turn to see how it
% changes the colors of your current slide theme.

%\usecolortheme{albatross}
%\usecolortheme{beaver}
%\usecolortheme{beetle}
%\usecolortheme{crane}
%\usecolortheme{dolphin}
%\usecolortheme{dove}
%\usecolortheme{fly}
%\usecolortheme{lily}
%\usecolortheme{orchid}
%\usecolortheme{rose}
%\usecolortheme{seagull}
%\usecolortheme{seahorse}
%\usecolortheme{whale}
%\usecolortheme{wolverine}

\addtobeamertemplate{navigation symbols}{}{%
    \usebeamerfont{footline}%
    \usebeamercolor[fg]{footline}%
    \hspace{1em}%
    \insertframenumber/\inserttotalframenumber
}
%\setbeamertemplate{footline} % To remove the footer line in all slides uncomment this line
%\setbeamertemplate{footline}[page number] % To replace the footer line in all slides with a simple slide count uncomment this line

%\setbeamertemplate{navigation symbols}{} % To remove the navigation symbols from the bottom of all slides uncomment this line
}

\usepackage{graphicx} % Allows including images
\usepackage{booktabs} % Allows the use of \toprule, \midrule and \bottomrule in tables
\usepackage{epstopdf}
\usepackage[backend=bibtex, style=numeric-comp, isbn=false, url=false, hyperref=true]{biblatex}
\usepackage{bibentry}
\usepackage[algoruled]{algorithm2e}
\setbeamertemplate{bibliography item}{\insertbiblabel}

\renewcommand*{\bibfont}{\tiny}

%----------------------------------------------------------------------------------------
%	TITLE PAGE
%----------------------------------------------------------------------------------------

\title[]{Haplotype Inference via Boolean Satisfiability and Integer Linear Programming} % The short title appears at the bottom of every slide, the full title is only on the title page

\author{Jarkko Savela} % Your name
\institute{University of Helsinki} % Your institution as it will appear on the bottom of every slide, may be shorthand to save space
\date{\today} % Date, can be changed to a custom date
\bibliography{paper}
\begin{document}

\begin{frame}
\titlepage % Print the title page as the first slide
\end{frame}

\begin{frame}
\begin{itemize}
\item \fullcite{DBLP:conf/alcob/BrownZG20}
\end{itemize}
\end{frame}

\begin{frame}
\frametitle{Background}
\begin{itemize}
\item ILP has been very successful in computational biology~\cite{gusfield2019integer}
\item BUT ILP is ineffective for certain problems
\item SAT has been successfully applied to certain computational biology problems such as haplotyping by pure parsimony~\cite{DBLP:conf/aaai/LynceM06, DBLP:journals/anor/GracaMLO11, DBLP:journals/jda/JagerCZ16}
\item Try other techniques (such as SAT) on problems ILP cannot handle
\end{itemize}
\end{frame}

\begin{frame}
\frametitle{Article contents}
\begin{itemize}
\item Empirical comparison of SAT-encodings and ILP formulations of 4 problems from computational biology
\begin{enumerate}
\item Protein folding under HP model
\item Transforming gene orders by reversals
\item Computing the history bound
\item \textbf{Haplotyping by pure parsimony}
\end{enumerate}
\item SAT solver used is pLingeling and Glucose, ILP solver Gurobi
\item ILP was better for problem 1 with some exceptions
\item SAT was superior in solving problems 2, 3 and 4 (wrt runtime and avoiding extreme behavior of ILP solvers)
\item SAT was able to solve instances of problem 4 that ILP could not solve
\end{itemize}
\end{frame}

\section{Haplotype Inference by Pure Parsimony}
\begin{frame}
\frametitle{Haplotype Inference Problem}
\begin{itemize}
\item Genetic material composed of 4 bases (A, T, C, G)
\item Humans are diploid: One copy of each chromosome inherited from each parent
\item[$\rightarrow$] Two copies of each gene
\item Assume two different alleles (0 and 1)
\item Genotype represented as a string over $\{0,1,2\}$
\begin{itemize}
\item 0: both alleles are 0
\item 1: both alleles are 1
\item 2: one allele 0 and other allele 1
\end{itemize}
\item haplotype is string over $\{0,1\}$
\item maternal haplotype $+$ paternal haplotype $=$ genotype
\end{itemize}
\end{frame}

\begin{frame}
\begin{itemize}
\item Given a genotype, what are the haplotypes resolving into this genotype?
\item Genotype 0021201 can be resolved into different pairs of haplotypes
\begin{itemize}
\item $\begin{matrix}
0001001  \\\hline
0011101 
\end{matrix}$
\item $\begin{matrix}
0011001 \\\hline
0001101 
\end{matrix}$
\end{itemize}
\item Which one is the correct pair of haplotypes?
\item Pure parsimony
\item Experimentally validated to give relatively accurate results~\cite{DBLP:journals/bioinformatics/WangX03a}
\end{itemize}
\end{frame}

\begin{frame}
\begin{itemize}
\item $\mathcal{G} = \{g_i\}$ a set of $n$ genotypes of length $m$.
\end{itemize}
\begin{block}{Haplotype Inference by Pure Parsimony (HIPP)}
For each $g_i$ find a pair of haplotypes $(h_i, h'_i)$ that resolves into $g_i$ such that the total number of used haplotypes is minimized.
\end{block}
\begin{itemize}
\item Can be solved by 
\begin{itemize}
\item Customized branch-and-bound algorithms
\item Reductions to SAT, ILP, ASP, PBO \dots
\end{itemize}
\end{itemize}
\end{frame}

\section{SAT Encoding}

\begin{frame}
\frametitle{HIPP-SAT}
\begin{itemize}
\item Formulated by Lynce and Marques-Silva~\cite{DBLP:conf/aaai/LynceM06}
\item High-level structure
\begin{itemize}
\item Boolean variables to represent haplotype strings
\item Selector variables to select a haplotype pair for each genotype
\end{itemize}
\item Formula $\phi (\mathcal{G}, r)$ is SAT iff exists a set of $r$ haplotypes resolving all genotypes in $\mathcal{G}$
\item Find the least $r$ such that $\phi (\mathcal{G}, r)$ is satisfied
\end{itemize}
\end{frame}

\begin{frame}
\begin{itemize}
\item $h_{k,j}$ represents $j$'th entry of haplotype $h_k$
$$ h_k = h_{k,1} h_{k,2} \dots h_{k,m-1} h_{k,m} $$
\item $s^a_{k_1,i}$ and $s^b_{k_2, i}$ select the pair of haplotypes for genotype $i$
\item If $s^a_{k_1,i}$ and $s^b_{k_2, i}$ evaluate to true, the haplotype pair for $g_i$ is $(h_{k_1}, h_{k_2})$
\end{itemize}
\end{frame}

\begin{frame}
\frametitle{Constraints}
\begin{itemize}
\item[1] The selector variables $s^a_{k_1,i}$ and $s^b_{k_2, i}$ must choose a unique pair of haplotypes
$$ \sum_{1\leq k \leq r} s^a_{k,i} = 1 \text{ and } \sum_{1\leq k \leq r} s^b_{k,i} = 1 $$
\item[2] The selected haplotype pair must resolve into the correct genotype
\end{itemize}
\end{frame}

\begin{frame}
\begin{itemize}
\item Constraint 1 implemented using a \emph{1-bit adder circuit}
\item Uses variables $v^a_{k, i}$
\end{itemize}
$$ \begin{array}{c|c|c|c|c|c|c}
v^a_{1,i} & v^a_{2,i} & v^a_{3,i} & \dots & v^a_{r-1, i} & v^a_{r, i} & v^a_{r+1, i} \\\hline
s^a_{1, i} & s^a_{2, i} & s^a_{3, i} & \dots & s^a_{r-1, i} & s^a_{r, i} & 
\end{array} $$
\begin{itemize}
\item Constraints
\begin{enumerate}
\item $v^a_{1, i} \leftrightarrow s^a_{1,i}$
\item $\neg v^a_{k,i} \vee \neg s^a_{k,i}$
\item $ v^a_{k+1,i} \leftrightarrow (s^a_{k,i} \vee v^a_{k,i})$
\item $ v^a_{r+1,i} = 1$
\end{enumerate}
\end{itemize}
\end{frame}

% TODO: Picture for next slide?
\begin{frame}
\begin{itemize}
\item Three cases depending on $g_{i,j}$ ($j$'th entry of genotype $i$)
\item If $g_{i,j} = 0$, both haplotypes must have 0 at index $j$
$$ s^a_{k,i} \rightarrow \neg h_{k,j}\quad \text{ and }\quad s^b_{k,i} \rightarrow \neg h_{k,j} $$ 
\item If $g_{i,j} = 1$, both haplotypes must have 1 at index $j$
$$ s^a_{k,i} \rightarrow h_{k,j}\quad \text{ and }\quad s^b_{k,i} \rightarrow h_{k,j} $$
\item If $g_{i,j} = 2$, the haplotypes must have different entries at $j$
$$ (g^a_{i,j}\vee g^b_{i,j}) \wedge (\neg g^a_{i,j} \vee \neg g^b_{i,j}) $$
$$ s^a_{k,i} \rightarrow (h_{k,j} \leftrightarrow g^a_{i,j}) $$
$$ s^b_{k,i} \rightarrow (h_{k,j} \leftrightarrow g^b_{i,j}) $$
\end{itemize}
\end{frame}

\begin{frame}
\frametitle{Optimizations}
\begin{itemize}
\item Computing lower bounds for $r$
\item[$\rightarrow$] Can skip evaluating $\phi (\mathcal{G}, r)$ for $r < lb$
\item Symmetry-breaking by ordering constraints
\item lexicographically order the haplotypes $h_1$, $h_2$,\dots
\end{itemize}
\end{frame}

\section{ILP encodings}

\begin{frame}
\frametitle{RTIP}
\begin{itemize}
\item Formulated by Gusfield~\cite{DBLP:conf/cpm/Gusfield03}
\item Idea: unfold potential haplotype pairs and use ILP to choose a minimum-size set to explain genotypes
\item $\mathcal{G} = \{g_i\}$ set of $n$ genotypes of length $m$
\item Assign binary variable $x_i$ for each potential haplotype
\item Haplotype $h_i$ is used iff $x_i = 1$
\item Binary variable $y_{i,j}$ for each haplotype pair $p_j$ of $g_i$
\item $y_{i,j} = 1$ iff haplotype pair $p_j$ is used to explain genotype $g_i$ 
\item Goal: Minimize $\displaystyle \sum_i x_i$
\end{itemize}
\end{frame}

\begin{frame}
\frametitle{Constraints}
\begin{enumerate}
\item Every genotype is explained by exactly one haplotype pair
$$ \sum_{j} y_{i,j} = 1 $$
\item If haplotype pair $p_j=(h_k, h_l)$ is used, then the haplotypes in the pair are used
$$ y_{i,j} - x_k \leq 0 $$
$$ y_{i,j} - x_l \leq 0 $$
\item The $x$ and $y$ variables have binary domain
$$ x_l\in \{0, 1\} $$
$$ y_{i,j}\in \{0, 1\} $$
\end{enumerate}
\end{frame}

\begin{frame}
\frametitle{Constructing the encoding}
\begin{itemize}
\item For each genotype $g_i\in \mathcal{G}$:
\item Generate all possible pairs of haplotypes $p_1, \dots , p_{k_i}$ explaining $g_i$
\item For each pair $p_j$, include variable $y_{i,j}$
\item For each haplotype $h_l$ not encountered earlier, include variable $x_l$
\item If $g_i$ has $b$ heterozygous sites, then there are $2^{b-1}$ distinct pairs that explain $g_i$
\item[$\rightarrow$] Exponential size
\end{itemize}
\end{frame}


\begin{frame}
\frametitle{PolyIP}
\begin{itemize}
\item Formulated by Brown and Harrower~\cite{DBLP:journals/tcbb/BrownH06}
\item Entries of genotype vectors non-standard
\begin{itemize}
\item 0: homozygous with allele 0
\item 1: heterozygous
\item 2: homozygous with allele 1
\end{itemize}
\item Now haplotype pair $(h, h')$ resolves genotype $g = h + h'$
\item Idea
\begin{itemize}
\item Structured representation of haplotypes
\item Machinery for counting distinct haplotypes
\end{itemize}
\end{itemize}
\end{frame}

\begin{frame}
\frametitle{Constraints}
\begin{itemize}
\item Variable $h_{i,j}$ is the $j$'th entry of haplotype $h_i$
\item Genotype $g_i$ is explained by haplotypes $h_{2i-1}$ and $h_{2i}$
$$ h_{2i-1, j} + h_{2i, j} = g_{i, j} \quad \forall j $$
\item Variable $d_{i,j}$ detects if $h_i$ and $h_j$ have a difference
$$ d_{i,j} \geq h_{i, k} - h_{j,k} \quad \forall k $$
$$ d_{i,j} \geq h_{j, k} - h_{i,k} \quad \forall k $$
\item If $d_{i,j}=1$ haplotypes $h_i$ and $h_j$ differ
\end{itemize}
\end{frame}

\begin{frame}
\begin{itemize}
\item Variable $x_i$ detects whether haplotype $h_i$ is the same as one of $h_1, \dots , h_{i-1}$
$$ x_{i} \geq 2 - i +  \sum_{j=1}^{i-1} d_{i,j} $$
\item If $x_i = 1$, then $h_i$ is a ``fresh'' entry
\item All variables are binary
$$ x_i \in \{0, 1\} $$
$$ h_{i,j} \in \{0, 1\} $$
$$ d_{i,j} \in \{0, 1\} $$
\end{itemize}
\end{frame}

\section{Comparison}
\begin{frame}
\frametitle{Comparison}
\begin{itemize}
\item Size of encoding
\begin{itemize}
\item RTIP exponential
\item HIPP-SAT and PolyIP polynomial
\end{itemize}
\item Representation of haplotypes
\begin{itemize}
\item RTIP uses atomic representation
\item HIPP-SAT and PolyIP use structured representation
\end{itemize}
\item RTIP fast to solve, but does not scale to large instance
\item PolyIP slower to solve, but scales better
\item[$\rightarrow$] HybridIP by Brown and Harrower
\item HIPP-SAT fastest and scales to largest instances
\end{itemize}
\end{frame}

\begin{frame}{}
\centering \LARGE \emph{Thank you for your attention!}
\end{frame}

\begin{frame}[allowframebreaks]
\frametitle{References}
\printbibliography
\end{frame}


\end{document} 