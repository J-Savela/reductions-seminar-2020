\documentclass[12pt,a4paper]{article}
%\usepackage[margin=1in]{geometry}

% Localization and encoding
\usepackage[utf8]{inputenc}
\usepackage[english]{babel}
\usepackage{csquotes}

% Math
\usepackage{amsthm}
\usepackage{amsmath}
\usepackage{amsfonts}

% Graphics
%\usepackage{rotating}
\usepackage{graphicx}
\graphicspath{ {./resources/images/} }
\usepackage{multirow}

% algorithms
%\usepackage{algorithm}
%\usepackage{algorithmic}
\usepackage[algoruled]{algorithm2e}

\usepackage{pdfpages}
\usepackage{multirow}
\usepackage{placeins}
\usepackage{hyperref}
\usepackage{caption}
\usepackage{tikz}
\usetikzlibrary{math}
\usetikzlibrary{calc}
\usetikzlibrary{arrows.meta}



% Counters
\newcounter{thmCounter}
\newcounter{defCounter}

% Theorem environments
\newtheorem{observation}{\bf Observation}
\newtheorem{theorem}{\bf Theorem}
\newtheorem{lemma}{\bf Lemma}
\newtheorem{prop}{\bf Proposition}
\newtheorem{rmk}{\bf Remark}
\newtheorem{definition}{\bf Definition}
\newtheorem{example}{\bf Example}
\newtheorem{corollary}{\bf Corollary}

\newcommand{\TODO}[1]{\textcolor{red}{#1}}



\sloppy
\title{Comparing SAT and ILP formulations for Computational and Systems Biology problems}
\date{2020 Autumn}
\author{Jarkko Savela}

\begin{document}
\pagenumbering{gobble}
\maketitle
\newpage
\tableofcontents
\newpage
\pagenumbering{arabic}

\section{Introduction}


\section{Boolean Satisfiability}
Boolean satisfiability~\cite{DBLP:series/faia/2009-185}, often referred to as the SAT problem, is the satisfiability problem of propositional logic.
That is, the problem is to find an assignment $s$ that satisfies the input formula $\phi$.
The formula $\phi$ is inductively constructed from propositional atoms (i.e., Boolean variables) and the usual connectives; conjunction $\wedge$, disjunction $\vee$, negation $\neg$, implication $\rightarrow$ and bi-implication $\leftrightarrow$.
The assignment $s$ gives each boolean variable in $\phi$ a truth value, 0 or 1.

In practice the input formula is usually restricted to be in \emph{conjunctive normal form} (CNF).
A formula $\phi$ is in CNF if it is a conjunction of \emph{clauses}
$$ \bigwedge_{i\in I} C_i $$
where each clause $C_i$ is a disjunction of \emph{literals}
$$ C_i = \bigvee_{j\in J_i} l_j $$
and each literal is either an atom $p$ or a negated atom $\neg p$.
CNF is a convenient standard format partly due to the fact that efficient polynomial-time algorithms exist for transforming arbitrary formulas into \emph{linear-size, equisatisfiable} CNF formulas~\cite{tseitin1983, DBLP:journals/jsc/PlaistedG86}.

Practically efficient algorithms for solving the SAT problem have been developed despite the fact that SAT is an NP-Complete problem~\cite{DBLP:conf/stoc/Cook71}, 
and SAT has become an importent problem solving technique with applications in model checking and verification, planning, scheduling, cryptography, computational biology et cetera.
The most efficient current SAT solvers are based on the \emph{conflict-driven clause learning} algorithm (CDCL) and use many optimizations and efficient data structures~\cite{DBLP:conf/iccad/SilvaS96, DBLP:journals/tc/Marques-SilvaS99, DBLP:conf/dac/MoskewiczMZZM01, DBLP:conf/aaai/GomesSK98, DBLP:journals/dam/GoldbergN07, ryan2004efficient, DBLP:conf/aaai/BayardoS97, DBLP:conf/sat/LewisSB05}.


\section{Integer Linear Programming}
Integer linear programming (ILP) is the problem of finding an optimal assignment to a set of integer variables subject to constraints in the form of linear inequalities~\cite{DBLP:books/ph/PapadimitriouS82}.
An ILP program thus consists of a vector of integer variables $\mathbf{x}$, a cost vector $\mathbf{c}$ as well as a matrix $A$ and vector $\mathbf{b}$ of coefficients with the goal of minimizing $\mathbf{c}^T \mathbf{x}$ under the constraints that $A\mathbf{x} \leq \mathbf{b}$.

ILP is also an NP-hard problem~\cite{DBLP:conf/coco/Karp72} and finds applications in planning, scheduling, computational biology and in many other domains.

The \emph{simplex algorithm} is a practically efficient (although worst-case exponential time) algorithm for solving general linear programs where the variables may take non-integral values as well.
ILPs can be solved via a \emph{branch-and-bound} procedure in which the simplex algorithm is used to solve the continuous relaxations of the program with or without the use of \emph{cutting planes}.
Branch-and-bound procedure utilizing cutting planes, i.e., \emph{branch-and-cut}, uses auxiliary constraints to prune away non-integral solutions~\cite{DBLP:books/ph/PapadimitriouS82}.


\section{Protein folding}


\section{Haplotype inference}


\bibliographystyle{plain}
\bibliography{paper.bib}

\end{document}
